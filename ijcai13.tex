%%%% ijcai11.tex

\typeout{IJCAI-13 Instructions for Authors}

% These are the instructions for authors for IJCAI-13.
% They are the same as the ones for IJCAI-11 with superficical wording
%   changes only.

\documentclass{article}
% The file ijcai13.sty is the style file for IJCAI-13 (same as ijcai07.sty).
\usepackage{ijcai13}

% Use the postscript times font!
\usepackage{times}

% the following package is optional:
%\usepackage{latexsym} 

\usepackage{amsthm}
\usepackage{amsmath}
\usepackage{amsfonts}

\newtheorem{theorem}{Theorem}
\newtheorem{definition}{Definition}
\newtheorem{corollary}{Corollary}
\newtheorem{lemma}{Lemma}
\newtheorem{proposition}{Proposition}

% Following comment is from ijcai97-submit.tex:
% The preparation of these files was supported by Schlumberger Palo Alto
% Research, AT\&T Bell Laboratories, and Morgan Kaufmann Publishers.
% Shirley Jowell, of Morgan Kaufmann Publishers, and Peter F.
% Patel-Schneider, of AT\&T Bell Laboratories collaborated on their
% preparation.

% These instructions can be modified and used in other conferences as long
% as credit to the authors and supporting agencies is retained, this notice
% is not changed, and further modification or reuse is not restricted.
% Neither Shirley Jowell nor Peter F. Patel-Schneider can be listed as
% contacts for providing assistance without their prior permission.

% To use for other conferences, change references to files and the
% conference appropriate and use other authors, contacts, publishers, and
% organizations.
% Also change the deadline and address for returning papers and the length and
% page charge instructions.
% Put where the files are available in the appropriate places.

\title{IJCAI--13 Formatting Instructions\thanks{These match the formatting instructions of IJCAI-07. The support of IJCAI, Inc. is acknowledged.}}
\author{Francesca Rossi \\
University of Padova\\
Italy \\
pcchair13@ijcai.org}

\begin{document}

\maketitle

\begin{abstract}
  The {\it IJCAI--13 Proceedings} will be printed from electronic
  manuscripts submitted by the authors. The electronic manuscript will
  also be included in the online version of the proceedings. This paper
  provides the style instructions.
\end{abstract}

\section{Introduction}

\section{General Test Game}

A \emph{test game} $G = (A, \mathbb B, p, v, m, t)$ is a 2-player Bayes game
between the tester and the test taker. The tester has a set of potential
problems $A = \{a_1, a_2, \ldots, a_n\}$ to test and the test taker has an
uncertain type (thus Bayes). The possible types are characterized by $\mathbb B
= \{B_1, B_2, \ldots, B_k\}$ and functions $p, v, m$. Set $B_l \subseteq A$ is
the unsolvable problem set of the $l$-th type test taker.  Function $p: \mathbb
B \rightarrow [0,1]$ characterizes the probability that a particular type test
taker occurs.  The memorize capacity function $m: \mathbb B \rightarrow \mathbb
N$ denotes how many problems a particular test taker can memorize and the value
function $v: \mathbb B \rightarrow \mathbb R^+$ denotes how much utility the
tester gets if that test taker failed the test (recall that the tester cannot
fail test takers who can solve all problems and he want to fail all others).
For simplicity, we write $p(B_l), m(B_l), v(B_l)$ as $p_l, m_l, v_l$ when the
context is clear.  During the test, $t$ out of $n$ problems will be tested and
a particular test taker will pass the test if each of those tested problems is
either not in the unsolvable problem set or memorized.  

The game only has two outcomes, pass or fail. The tester gets $0$ utility if a
test taker passes and he gets $v_l$ utility if a type-$l$ test taker fails. On
the type-$l$ test taker's side, he gets $0$ utility if he passes and $-v_l$ if
he fails. From the test taker's perspective, any utility function that has a
higher pass utility than fail utility will give him the same incentive to pass
the test at his best. We define them as $0, -v_l$ specifically for the zero-sum
property of the game.

We want to find the optimal strategy for the tester to maximize his utility
under Stackelberg settings. That is, the tester firstly reveals his strategy
about how $t$ problems are picked to test, then the test taker chooses the best
memorizing strategy. This corresponds to our confidential-free assumption where
test takers know what problems might be tested, their answers, and how likely
they are tested. The only constraint for test takers is that they cannot
memorize too many (greater than $m_l$) problem answers. 

In summary, 

\subsection{General LP Formulation}

Though general

\subsection{Hardness on Test Size}

In order to formally characterize \emph{test game}'s hardness, define

\begin{definition}[\textsc{Optimal Test Strategy} Problem]
Given a test game $G$ and a value $u$, the \textsc{Optimal Test Strategy}
problem is to decide whether the tester has a Stackelberg strategy with at
least $u$ utility.
\end{definition}

TO BE CONTINUED WITH Vertex Cover

\subsection{Hardness on Memorize Capacity}


\begin{theorem}
Even if test size is $2$, \textsc{Optimal Test Strategy} is coNP-hard when
memorize capacity is non-constant.
\end{theorem}

\begin{proof}
We reduce \textsc{Independent Set} instances to test games with test size $2$
and ask the complementary question: is $u$ the highest utility that the tester
can get. That question has a yes answer if and only if the dual minimax LP
[reference] has a feasible solution with objective at most $u$. For a graph $G
= (V, E)$, let $V$ be our test problem set. Add $|E|+|V|+2$ types of test
takers as follows:

\begin{itemize}

\item For each edge $e = (v_i, v_j) \in E$, add one type of test takers $l_e$
with tester utility equals $1$ if they fail and $0$ if they pass. This type of
test takers cannot solve only two problems, $v_i, v_j$. The number of problems
they can memorize is 0.

\item For each vertex $v$, we add one type of test takers $l_v$ with tester utility
equals $d_G-d(v)$ if they fail and 0 if they pass. Here $d_G$ is the maximum
vertex degree in $G$ and $d(v)$ is the degree of vertex $v$. This type of test
takers cannot solve only one problem $v$ and the number of problems they can
memorize is 0.

\item Add one auxiliary type of test takers $l_A$ with tester utility equals
$a \cdot \varepsilon$ if they fail and $0$ if they pass where $a =
\binom{|V|}{2}-|E|-\binom{k}{2}$. This type of test takers cannot solve any
problems but they can memorize $2$.

\item Finally, add one type of test takers $l_K$ with tester utility equals
$u^{l_K} = \varepsilon$ if they fail and $0$ if they pass. This type of test
takers cannot solve any problems but they can memorize $k$ of them.

\end{itemize}

Then we show that deciding whether $u = (2 d_G + a\cdot \varepsilon)/L$ is an
upper-bound for the tester's utility, or equivalently whether we can find a
dual solution with objective as low as $u = (2 d_G + a\cdot \varepsilon)/L$ in
our dual minimax LP, is equivalent to finding a size-$k$ independent set in
$G$. Recall that a dual solution is a strategy for test takers to memorize
problems. If no one memorizes, the tester's utility $U_r$ to test a problem set
$r$ is $U_r = (2d_G+a\cdot\varepsilon + \varepsilon) /L$ if $r \notin E$ and
$U_r = (2d_G-1+a\cdot\varepsilon + \varepsilon)/L$ if $r \in E$. To lower
the objective, test takers $l_A, l_K$ should memorize problem sets that bring the
maximum of $U_r$ down.  No matter what the strategy is (as long as they only
memorize unsolvable problems), $\sum_r U_r$ is a constant. And because $U_r$
for $r \notin E$ is much larger than $U_r$ for $r \in E$ in terms of
$\varepsilon$, it's better to only decrease $U_r$ for $r \notin E$. The only
way to do that is to let $l_A$ test takers only memorize pairs of problems
that are not edges and let $l_K$ test takers only memorize size-$k$ indepedent
sets of problems.  If there's a size-$k$ independent set, let $l_K$ test takers
memorize one size-$k$ independent set all the time and let $l_A$ test takers
memorize all pairs of problems that are not covered by that independent set with
uniform probability. That will make $U_r = (2d_G+a\cdot\varepsilon)/L$ for $r
\notin E$ and $U_r = (2d_G-1+a\cdot\varepsilon+\varepsilon)/L$ for $r \in E$.
This is the best they can achieve and this can only be achieved when a size-$k$
independent set exists in $G$.

\end{proof}

\section{One-Problem Test Game}

\section{Experiment}

\section{Conclusion}

\section*{Acknowledgments}

The preparation of these instructions and the \LaTeX{} and Bib\TeX{}
files that implement them was supported by Schlumberger Palo Alto
Research, AT\&T Bell Laboratories, and Morgan Kaufmann Publishers.
Preparation of the Microsoft Word file was supported by IJCAI.  An
early version of this document was created by Shirley Jowell and Peter
F. Patel-Schneider.  It was subsequently modified by Jennifer
Ballentine and Thomas Dean, Bernhard Nebel, and Daniel Pagenstecher.
These instructions are the same as the ones for IJCAI--05, prepared by
Kurt Steinkraus, Massachusetts Institute of Technology, Computer
Science and Artificial Intelligence Lab.

\appendix

\section{\LaTeX{} and Word Style Files}\label{stylefiles}

The \LaTeX{} and Word style files are available on the IJCAI--13
website, {\tt http://www.ijcai-13.org/}.
These style files implement the formatting instructions in this
document.

The \LaTeX{} files are {\tt ijcai13.sty} and {\tt ijcai13.tex}, and
the Bib\TeX{} files are {\tt named.bst} and {\tt ijcai13.bib}. The
\LaTeX{} style file is for version 2e of \LaTeX{}, and the Bib\TeX{}
style file is for version 0.99c of Bib\TeX{} ({\em not} version
0.98i). The {\tt ijcai13.sty} file is the same as the {\tt
ijcai07.sty} file used for IJCAI--07.

The Microsoft Word style file consists of a single file, {\tt
ijcai13.doc}. This template is the same as the one used for
IJCAI--07.

These Microsoft Word and \LaTeX{} files contain the source of the
present document and may serve as a formatting sample.  

Further information on using these styles for the preparation of
papers for IJCAI--13 can be obtained by contacting {\tt
pcchair13@ijcai.org}.

%% The file named.bst is a bibliography style file for BibTeX 0.99c
\bibliographystyle{named}
\bibliography{ijcai13}

\end{document}

